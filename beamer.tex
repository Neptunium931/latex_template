\documentclass{beamer}
\usetheme{Madrid}

\date{2023}

\title[About Beamer] %optional
{About the Beamer class in presentation making}

\subtitle{A short story}

\author[Arthur, Doe] % (optional, for multiple authors)
{A.~B.~Arthur\inst{1} \and J.~Doe\inst{2}}

\institute[VFU] % (optional)
{
  \inst{1}%
  Faculty of Physics\\
  Very Famous University
  \and
  \inst{2}%
  Faculty of Chemistry\\
  Very Famous University
}

\date[02/05/2024] % (optional)
{Very Large Conference, April 2021}
\AtBeginSection[]{
  \begin{frame}
    \begin{center}{\Large Plan }\end{center}
    %%% affiche en début de chaque section, les noms de sections et
    %%% noms de sous-sections de la section en cours.
    \tableofcontents[currentsection,hideothersubsections]
  \end{frame} 
}
\begin{document}

\frame{\titlepage}

\begin{frame}
  \frametitle{Sample frame title}
  This is some text in the first frame. \pause

  This is some text in the first frame. 

  This is some text in the first frame.
\end{frame}

\begin{frame}
  \frametitle{Sample frame title}

  In this slide, some important text will be
  \alert{highlighted} because it's important.
  Please, don't abuse it.

  \begin{block}{Remark}
    Sample text
  \end{block}

  \begin{alertblock}{Important theorem}
    Sample text in red box
  \end{alertblock}

  \begin{examples}
    Sample text in green box. The title of the block is ``Examples".
  \end{examples}
\end{frame}

\section[titre court1]{titre long long long 1}
\begin{frame}
  \frametitle{frame1}
\end{frame}

\section[titre court2]{titre long long long 2}

\subsection[titre court1]{titre long long long 1}
\begin{frame}
  \frametitle{frame2}
\end{frame}

\subsection[titre court2]{titre long long long 2}
\begin{frame}
  \frametitle{frame3}
\end{frame}

\subsection[titre court3]{titre long long long 3}
\begin{frame}
  \frametitle{frame4}
\end{frame}

\section[titre court3]{titre long long long 3}
\begin{frame}
  \frametitle{frame5}
\end{frame}
\begin{frame}
  % \pause: Hides everything after the \pause command until the next step.
  % \only<slide>: Displays its content only on the specified slide(s).
  % \onslide<slide>: Displays its content only on the specified slide(s), with options for controlling appearance.
  % \visible<slide>: Makes its content visible on the specified slide(s), but the content is still present on other slides.
  % \uncover<slide>: Makes its content visible on the specified slide(s), with options for controlling appearance.
  \frametitle{Combining Different Commands}
  This is the first part of the slide.\newline
  \pause
  This is the second part of the slide.\\
  \only<3>{This is the third part of the slide.\\}
  \onslide<4>{This is the fourth part of the slide.\\}
  \visible<5>{This is the fifth part of the slide.\\}
  \uncover<6>{This is the sixth part of the slide.\\}
\end{frame}
\end{document}
